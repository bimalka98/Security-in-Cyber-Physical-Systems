\documentclass[11pt,letterpaper]{article}

\usepackage[letterpaper,margin=0.8in,nohead]{geometry}

\usepackage[colorlinks]{hyperref}
\usepackage{url}
\usepackage{breakurl}

\hypersetup{
	colorlinks,
	linkcolor={red},
	citecolor={red},
	urlcolor={blue}
}

\usepackage{verbatim}
\usepackage{fancyvrb}
\usepackage{scrextend}
\usepackage{enumitem}
\usepackage{url}
\usepackage{tabularx}

\usepackage{caption}
\usepackage{graphicx}
\usepackage{float}
\usepackage{subcaption}

\usepackage{changepage}   % for the adjustwidth environment

\newenvironment{answer}{\em \color{blue} \begin{adjustwidth}{1cm}{1cm}}{\end{adjustwidth}}

% math
\usepackage{amsthm,amsmath}
\usepackage{amssymb}
\usepackage{amsfonts}

\newcommand{\mc}[1]{\mathcal{#1}}	% Mechanisms / Algorithms
\newcommand{\rv}[1]{\mathbf{#1}}    % Random variable

\newcommand{\pr}[1]{\mathrm{Pr}\{#1\}} % Probability

\newtheorem{corollary}{\bf Corollary}%[theorem]
\newtheorem{lemma}{\bf Lemma}%[theorem]
\newtheorem{definition}{\bf Definition}%[section]

\newtheorem{observation}{\bf Observation}%[theorem]

% load cleveref last!
\usepackage[capitalise]{cleveref}


\begin{document}
	
	\title{EN4720: Security in Cyber-Physical Systems \\ Exercise --- Encryption and Hashing}
	
	%% This is an individual assignment!!
	%% TODO: put your name and index number here here!
	\author{ \textcolor{blue}{Name: Thalagala B. P.} \\ \textcolor{blue}{Index No: 180631J}}
	
	\maketitle
	
	\begin{center}
		\color{red}\bf This is an individual exercise! \\ Due Date: 21 April 2022 by 11.59 PM
	\end{center}
	
	% \subsection*{Problem 1 - Block Cipher Modes of Operation}
	
	% Critically analyze block ciper modes of operation and fill the following table. You may search on the internet/refer to existing literature and discuss among friends to answer this question.  But plagiarism will not be tolerated! Make sure you add the relevant citations. One citation is given to you as an example below.
	% \vspace{2mm}
	
	% \noindent ``Recommendation for block cipher modes of operation'' published by NIST is a good resource to get started with~\cite{dworkin2001recommendation}.
	
	% \begin{table}[htbp]
		%     \caption{Comparison of modes of operation
			%     }
		%     \begin{tabularx}{\columnwidth}{|p{2cm}|X|X|X|}
			%         \hline
			%         \textbf{Mode} & \textbf{Pros}       & \textbf{Cons} & \textbf{Usage examples}  \\
			%         \hline
			%         ECB & 
			%         Pro 1 | Pro 2 | Pro 3 & 
			%         Con 1 | Con 2 | Con 3 & 
			%         Example 1 | Example 2 | Example 3 \\ \hline
			
			%         \hline
			%         CBC & 
			%         Pro 1 | Pro 2 | Pro 3 & 
			%         Con 1 | Con 2 | Con 3 & 
			%         Example 1 | Example 2 | Example 3 \\ \hline
			
			%         \hline
			%         CFB & 
			%         Pro 1 | Pro 2 | Pro 3 & 
			%         Con 1 | Con 2 | Con 3 & 
			%         Example 1 | Example 2 | Example 3 \\ \hline
			
			%         \hline
			%         OFB & 
			%         Pro 1 | Pro 2 | Pro 3 & 
			%         Con 1 | Con 2 | Con 3 & 
			%         Example 1 | Example 2 | Example 3 \\ \hline
			
			%         \hline
			%         CTR & 
			%         Pro 1 | Pro 2 | Pro 3 & 
			%         Con 1 | Con 2 | Con 3 & 
			%         Example 1 | Example 2 | Example 3 \\ \hline
			%     \end{tabularx}
		% \end{table}
	
	\subsection*{Problem 1 - Symmetric Encryption \& Key Exchange}
	
	Alice and Bob are using the Diffie-Hellman protocol as shown in Figure~\ref{fig:dh-protocol}. They want to extend their key exchange mechanism to include their friend Charlie.
	
	\begin{figure}[h]
		\centering
		\includegraphics[width=0.7\columnwidth]{images/dh-protocol.jpg}
		\caption{Diffie Hellman Protocol} \label{fig:dh-protocol}
	\end{figure}
\pagebreak
	\begin{enumerate}
		\item Design an extension of the Diffie-Hellman protocol to allow secure key exchange between the three parties.\\
		
		\begin{answer}
			
			The below method can be used to extend the DH protocol to share a common key between three parties. The method is graphically represented in the Figure \ref{fig:dh-protocol-extension} as well. The answer was adopted from \url{https://crypto.stackexchange.com/a/1027}
			
			\begin{figure}[H]
				\centering
				\includegraphics[width=0.9\columnwidth]{images/3-parties-1-key.png}
				\caption{Extended Diffie-Hellman Protocol for three parties to share a common key} \label{fig:dh-protocol-extension}
			\end{figure}
		
		\begin{enumerate}
			\item Alice, Bob and Charlie each picks a random number from the chosen multiplicative group ($\mathbb{Z}_p$). Let those numbers be a, b and c respectively, and Alice, Bob and Charlie are in a loop as shown in the Figure \ref{fig:dh-protocol-extension}.
			
			\item Initially each party computes $Init_{val} = g^i ~mod~ p$ (\textbf{A1, B1, C1} in the Figure \ref{fig:dh-protocol-extension}), for their selected $i$ where $i ~\in \{a, b, c\}$ and passes to the next person in the loop.
			
			\item Upon receiving an $Init_{val}$ from the previous party of the loop, each party again computes $Inter_{val} = {Init_{val}}^i ~mod~ p$, (\textbf{A2, B2, C2} in the Figure \ref{fig:dh-protocol-extension}) and passes to the next person in the loop.
			
			\item Finally, upon receiving the $Inter_{val}$ from the previous party, each party calculates $Key_{val} = {Inter_{val}}^i ~mod~ p$ (\textbf{Key} in the Figure \ref{fig:dh-protocol-extension}), which becomes the common key between the three parties.
		\end{enumerate}
	
	The key will take the form $g^{abc} ~mod ~p$ after the above described process and will be a common key that can be used among the three parties for information encryption and decryption.
		
		\end{answer}
	\pagebreak
		\item Alice, Bob, and Charlie are planning to use $(p, g) = (23, 9)$ to agree on a key for a Caesar cipher. Assume that the Caesar cipher is right shifting the number of letters equal to the key to produce the ciphertext (e.g., if the key is 5 and the plaintext letter is U, it will be replaced by Z in the ciphertext). Assuming 4, 7, and 13 to be secret exponents of Alice, Bob, and Charlie respectively, find the common key that they need to use in Caesar's cipher.
		\begin{answer}
			\[
			\begin{split}
				Key &= g^{abc} ~mod ~p\\
				& = 9^{4 \times 7 \times 13} ~mod ~ 23\\
				& = 9
			\end{split}
			\]
		\end{answer}
		\item Derive the encrypted message for the message ``Hello" using the above encryption key.  Show all the major steps in deriving the answer.
		\begin{answer}		
		\begin{enumerate}
			\item Since the key is 9, each letter in the original message is right shifted by 9 letters to produce the Caesar cipher.
			\item  Assume there is no requirement to preserve the case sensitivity of the message. Then the message is ``HELLO''
			\item Now for each letter in the word, replace it by the letter 9 slots to its right. Then `H' is replaced by `Q', `E' is replaced by `N', `L' is replaced by `U' and `O' is replaced by `X'.
			\item  Then the ciphertext will be ``QNUUX"".
		\end{enumerate}
		\end{answer}
	\end{enumerate}
	
	\newpage 
	
	\subsection*{Problem 2 - Symmetric Encryption - Security Analysis}
	For this part of the exercise you are required to install \href{https://www.cryptool.org/en/ct1/}{CrypTool}. CrypTool is a software tool that comes with several encryption techniques.  
	
	\subsubsection*{Instructions}
	\begin{enumerate}
		\item Create a text file named \textbf{``sampletext.txt"} in your computer. Add the text ``EN4720: This is a sample text file for encryption." to the text file and save it.
		\item Open the \textbf{``sampletext.txt"} file with CrypTool software.
		\item Encrypt your message with RC2 encryption following \textbf{Encrypt/Decrypt \textgreater Symmetric (modern) \textgreater RC2}. Choose key size of 8 bits and add some random hexadecimal key to encrypt your text file. (Add screenshots)
		\item Run brute force analysis of the encrypted message by following \textbf{Analysis \textgreater Symmetric Encryption (modern) \textgreater RC2}. (Add screenshots)
		\item Repeat steps 3,4 for the key sizes 16, 24, 64, and 128 bits.
		\item Answer the questions given below.
	\end{enumerate}
	\pagebreak
	\subsubsection*{Questions}
	
	\begin{enumerate}
		\item Add screenshots requested in Step \#3.
		\begin{answer}
			\begin{figure}[H]
				\centering
				\includegraphics[width=0.6\columnwidth]{images/p2_q1_1}
				\caption{Open the \textit{sampletext.txt} file using the CrypTool software} \label{fig:enc-step-1}
			\end{figure}
		
			`\begin{figure}[H]
				\centering
				\includegraphics[width=0.6\columnwidth]{images/p2_q1_2}
				\caption{Enter the 8 bit Hexadecimal key for encryption} \label{fig:enc-step-2}
			\end{figure}
		
			\begin{figure}[H]
				\centering
				\includegraphics[width=0.6\columnwidth]{images/p2_q1_3}
				\caption{Encrypted text using the selected key} \label{fig:enc-step-3}
			\end{figure}
		\end{answer}
		
		\item Add screenshots requested in Step \#4.
		\begin{answer}
			\begin{figure}[H]
				\centering
				\includegraphics[width=0.6\columnwidth]{images/p2_q2_1}
				\caption{Configure the Brute Force analysis of RC2} \label{fig:dec-step-1}
			\end{figure}
			
			`\begin{figure}[H]
				\centering
				\includegraphics[width=0.6\columnwidth]{images/p2_q2_2}
				\caption{All possible keys obtained through the Brute Force attack} \label{fig:dec-step-2}
			\end{figure}
			
			\begin{figure}[H]
				\centering
				\includegraphics[width=0.6\columnwidth]{images/p2_q2_3}
				\caption{Decrypted message} \label{fig:dec-step-3}
			\end{figure}
		\end{answer}
	
	
		\item Complete Table~\ref{tab:time-taken-to-bruteforce}.
		\begin{table}[h!]
			\caption{Time taken to brute force} \label{tab:time-taken-to-bruteforce}
			\begin{tabularx}{\columnwidth}{|p{4cm}|X|X|}
				\hline
				\textbf{Key length (bits)} & \textbf{Time to brute force} & \textbf{Maximum number of keys} \\
				\hline
				% 9F
				8 &  $<$ 1 s& 33\\\hline
				
				\hline
				% 8A2D
				16 & $<$ 1 s & 1 \\\hline
				
				\hline
				24 & 1 min 34 s & 1\\ \hline
				
				\hline
				64 & $3.8\times10^6$ years& None \\ \hline
				
				\hline
				128 &$7.5\times10^{25}$ years& None\\ \hline
				
			\end{tabularx}
		\end{table}
		\item What is the impact of key length on time to decrypt a message using brute force?
		\begin{answer}
			When the key length is increased, time to decrypt the message grows drastically
		\end{answer}
		\item What could be done to diminish the amount of time required to perform the attack?
		\begin{answer}
			\begin{itemize}
				\item Guess some of the characters or enter the known parts in the key and fix them, rather than letting computer to perform every possible combination of the characters. This limits the search space of keys.
				
				\item Carry out the decryption in parallel using several computers.
			\end{itemize}
		\end{answer}
		
	\end{enumerate}
	\pagebreak
	\subsection*{Problem 3 - Public Key Encryption}
	
	You are given that $n=65$ and $\phi(n)=48$ for a certain RSA scheme.
	\begin{enumerate}
		\item Calculate the private key $d$ assuming the public key $e=7$.
		
		\begin{answer}
			$d$, takes the form $1 = e*d~mod ~\phi = 7*d~mod ~48$ and it can be calculated using the Extended Euclidean Algorithm. The steps are given below.
			
			\begin{enumerate}
				\item Euclidean Algorithm
				\[
				\begin{split}
					48 &= 6*\left(7\right) + 6\\
					7 &= 1*\left(6\right) + 1
				\end{split}
				\]
				
				\item Back Substitution
				\[
				\begin{split}
					1 &= 7 - 1*\left(6\right)\\
					1 &= 7 - 1*\left(48 - 6*\left(7\right) \right) \\
					1 &= 7*\left(7\right) - 1*\left(48\right)\\
				\end{split}
				\]
				
				\item The coefficient in front of the $e = 7$, is the required $d$. Therefore $d = 7$.
			\end{enumerate}					
			
		\end{answer}
		
		\item Show all major steps of encryption and decryption using textbook RSA for the message given below. Specifically, derive the ciphertext and show that the plaintext can be recovered at the receiver after decrypting.\\
		$m=$ last digit of your index number (e.g., if the index number is 180234N, $m=4$). If the last digital is 0 (zero) or 1 (one), use $m=5$. 
		\begin{answer}					
		\begin{enumerate}
			\item Encryption for $m=5$ since last digit of index number is 1. 
			\[
			\begin{split}
				ciphertext &= m^e ~mod ~n\\
				&= 5^7~mod~65 \\
				&=60 ~ \because (5^7 = 1201*65 + 60)
			\end{split}
			\]
			
			\item Decryption
			\[
			\begin{split}
				plaintext &= c^d ~mod~65\\
				&=60^7~mod~65\\
				&= \left\{ \left( 60^2~mod~65 \right)* \left( 60^2~mod~65 \right)* \left( 60^2~mod~65 \right) * \left( 60^1~mod~65 \right) \right\} ~mod ~65\\
				&= \left\{ \left( 25 \right)* \left( 25 \right)* \left( 25 \right) * \left( 60 \right) \right\} ~mod ~65\\
				&=5~ \because (25^3 * 60 = 14423*65 + 5)
			\end{split}
			\]

		\end{enumerate}
	$\therefore$ the plaintext can be recovered at the receiver after decrypting.
	\end{answer}

		\item Find the candidates for the primes $p,q$ that were used to generate this RSA scheme.
		
		\begin{answer}
			\begin{enumerate}
				\item $48$ can be written as, $48 = 12 * 4$
				\item $65$, can be written as, $65 = \left( 12 + 1 \right) * \left( 4 + 1\right)$
			\end{enumerate}
		$\therefore$ the used prime factors to generate $n$, are 13 and 5.
		\end{answer}
		\pagebreak
		\item Suppose that $c$ is the ciphertext that you find above for the message $m$. Show how an attacker can recover the plaintext message $m$ from the ciphertext $c$ assuming that he can request a decryption of a single ciphertext $c' \neq c$.
		
		\begin{answer}
			\begin{enumerate}
				\item The number $n=65~ and ~ e = 7$ is already a public knowledge. In order to decrypt any given message, attacker has to figure out only the private exponent $d$.
				
				\item Once the attacker has access to a ciphertext  $c'$ which corresponds to some plaintext $m'$, all he has to do is guess some private exponent, decrypt $c'$ using it and check it matches with $m'$.
				
				\item Once he found a match, it means he has recovered the private key $d$. Now he can decrypt any ciphertext which intends for the victim party.
			\end{enumerate}
			 
			 
			
		\end{answer}
		 
	\end{enumerate}
	\pagebreak
	\subsection*{Problem 4 - Hash Functions}
	
	{For this part of the exercise you are required to install \href{https://www.kali.org/get-kali/#kali-platforms}{Kali Linux}. Kali Linux comes preinstalled with several hashing utilities. These include \textbf{md5sum}, \textbf{sha1sum},
		\textbf{sha256sum}, and \textbf{sha512sum}. Follow the given instructions to complete the problem.}
	
	\subsubsection*{Instructions}
	\begin{enumerate}
		\item Create a new directory and a new text file ``\textbf{originalhashfile}'' in the directory using \textbf{nano originalhashfile} command. Add the text ``EN4720: Some text here'' in the created file and save the file using the hotkey \textbf{CTRL+X} followed by \textbf{Y} and \textbf{Enter}.
		
		\item Run the \textit{ls} command in the directory to show the file (add screenshot) and run the \textit{cat} command to show the contents of the created file (add screenshot).
		
		\item calculate the hash digest of the \textbf{originalhashfile} using these hash algorithms: MD5, SHA1, SHA256, SHA512 (add screenshots). Add your observations in Table~\ref{tab:hash-digest-originalhashfile}. You can use the following commands to get the results.
		\begin{itemize}
			\item md5sum \textless \textit{hashfile\_name}\textgreater
			\item sha1sum \textless \textit{hashfile\_name}\textgreater
			\item sha256sum \textless \textit{hashfile\_name}\textgreater
			\item sha512sum \textless \textit{hashfile\_name}\textgreater
		\end{itemize}
		
		\item Change the contents of the \textbf{originalhashfile} by adding your index at the end of the line (``EN4720: Some text here \textless\textit{index\_no}\textgreater'') and save it as ``\textbf{modifiedhashfile}'' in the same directory.
		
		\item Run the \textit{ls} command in the directory to show the files (add screenshot) and run the \textit{cat} command to show the contents of the modified file (add screenshot).
		
		\item Calculate the hash digest of the \textbf{modifiedhashfile} using these hash algorithms: MD5, SHA1, SHA256, SHA512 (add screenshots). Add your observations in Table~\ref{tab:hash-digest-modifiedhashfile}.
		
		\item Answer the questions given below.
		
	\end{enumerate}
	\pagebreak
	\subsubsection*{Questions}
	\begin{enumerate}
		
		\item Add screenshots requested in Step \#2.
		
		\begin{answer}
			\begin{figure}[H]
				\centering
				\includegraphics[width=0.7\columnwidth]{images/p3/s1}
				\caption{Output of the {\tt ls} command} \label{fig:pa-ls}
			\end{figure}
		
			\begin{figure}[H]
				\centering
				\includegraphics[width=0.7\columnwidth]{images/p3/s2}
				\caption{Output of the {\tt cat originalhashfile} command} \label{fig:pa-cat}
			\end{figure}
		\end{answer}
		

		
		\item Complete Table~\ref{tab:hash-digest-originalhashfile}. Add screenshots requested in Step \#3 below the table.
		
		% \begin{table}[h!]
			%     \caption{Hash digests of the original file
				%     } \label{tab:hash-digest-originalhashfile}
			%     \begin{tabularx}{\columnwidth}{|p{4cm}|X|}
				%         \hline
				%         \textbf{Hash Algorithm} & \textbf{Hash of the Original File} \\
				%         \hline
				%         MD5 & caf4d10c96b7b646e24ae72f366b4c7d \\\hline
				
				%         \hline
				%         SHA1 & e64870421f78773bf65437f515fd384170ce8839 \\\hline
				
				%         \hline
				%         SHA256 & cc544b593a0faa91f34774289a66b7400d1ce3fa36cfaec64a958228c5ead764 \\ \hline
				
				%         \hline
				%         SHA512 & cc7aebdfdf19a2f9882deb0ff1fc01a5be9f25cb2b916bd179b4ec88436113c5dc246\\&18487bed98fcc3d11574bafabde3a5c785eed6bccef496a7817d132d73a \\ \hline
				
				%     \end{tabularx}
			% \end{table}
		
		\begin{table}[h!]
			\caption{Hash digests of the original file
			} \label{tab:hash-digest-originalhashfile}
			\begin{tabularx}{\columnwidth}{|p{4cm}|X|}
				\hline
				\textbf{Hash Algorithm} & \textbf{Hash of the Original File} \\
				\hline
				MD5 & caf4d10c96b7b646e24ae72f366b4c7d \\\hline
				
				\hline
				SHA1 & e64870421f78773bf65437f515fd384170ce8839 \\\hline
				
				\hline
				SHA256 & cc544b593a0faa91f34774289a66b7400d1ce3fa36cfaec64a958228c5ead764  \\ \hline
				
				\hline
				SHA512 & cc7aebdfdf19a2f9882deb0ff1fc01a5be9f25cb2b916bd179b4ec88436113c5dc\\ & 24618487bed98fcc3d11574bafabde3a5c785eed6bccef496a7817d132d73a \\ \hline
				
			\end{tabularx}
		\end{table}

		\begin{figure}[H]
			\centering
			\includegraphics[width=0.7\columnwidth]{images/p3/s3}
			\caption{Output of the {\tt md5sum originalhashfile} command} \label{fig:md5sum}
		\end{figure}
	
		\begin{figure}[H]
			\centering
			\includegraphics[width=0.7\columnwidth]{images/p3/s4}
			\caption{Output of the {\tt sha1sum originalhashfile} command} \label{fig:sha1sum}
		\end{figure}
	
		\begin{figure}[H]
			\centering
			\includegraphics[width=0.7\columnwidth]{images/p3/s5}
			\caption{Output of the {\tt sha256sum originalhashfile} command} \label{fig:sha256sum}
		\end{figure}
	
		\begin{figure}[H]
			\centering
			\includegraphics[width=0.7\columnwidth]{images/p3/s6}
			\caption{Output of the {\tt sha512sum originalhashfile} command} \label{fig:sha512sum}
		\end{figure}
		
		\item Among the hash algorithms used, which one is the most cryptographically secure? Justify your answer.
		
		\begin{answer}
			{\tt sha512sum}. Security of a hash function depends on three factors: 1. Preimage resistance/ One-waynes, 2. Collision Resistance, and 3. Second Preimage resistance. When it comes to {\tt sha512sum} which is a SHA-2 hash, with the modern computing capabilities it is \textbf{hard to attack and break} the mentioned three properties of it.
		\end{answer}
		
		\item Why is MD5 not considered as a reliable hashing algorithm?
		
		\begin{answer}
			MD5 is an already broken hashing algorithm which is susceptible to collision attacks and birthday attacks. That is security experts have demonstrated methods, other than naive brute force attacks which can be used to break the hash even with limited computing capabilities.
		\end{answer}
		
		\item Which of the algorithms fall under the category of SHA-2?
		\begin{answer}
			{\tt sha256sum} and {\tt sha512sum}
		\end{answer}
		
		\item Add screenshots requested in Step \#5.
		
		\begin{answer}
			\begin{figure}[H]
				\centering
				\includegraphics[width=0.7\columnwidth]{images/p3/s7}
				\caption{Output of the {\tt ls} command} \label{fig:pa-ls2}
			\end{figure}
			
			\begin{figure}[H]
				\centering
				\includegraphics[width=0.7\columnwidth]{images/p3/s8}
				\caption{Output of the {\tt cat modifiedhashfile} command} \label{fig:pa-cat2}
			\end{figure}
		\end{answer}
		
		\item Complete Table~\ref{tab:hash-digest-modifiedhashfile}. Add screenshots requested in Step \#6 below the table.		
		
		\begin{table}[h]
			\caption{Hash digests of the modified file
			} \label{tab:hash-digest-modifiedhashfile}
			\begin{tabularx}{\columnwidth}{|p{4cm}|X|}
				\hline
				\textbf{Hash Algorithm} & \textbf{Hash of the Modified File} \\
				\hline
				MD5 & e335818d74506e6ef35419fea121b102 \\ \hline
				
				\hline
				SHA1 & b002d4e2319a4bce10a19779bcd37b6829c4ee1a \\ \hline
				
				\hline
				SHA256 & 2621e717d282129cac13b08f3f2c3b45a44af25f56b2b2d762fa553a6e9f54f8 \\ \hline
				
				\hline
				SHA512 & 0716eb7534e87ee353c2bf1ab16ca3a6747c19c1234426209e11d6c9ce1244e0cb\\&f2a4bc5a8175e8f26b4d457ab2a8f8fdf90568e8726b2d557a3780514a861c \\ \hline
				
			\end{tabularx}
		\end{table}
		
		\begin{figure}[H]
			\centering
			\includegraphics[width=0.7\columnwidth]{images/p3/s9}
			\caption{Output of the {\tt md5sum modifiedhashfile} command} \label{fig:md5sum2}
		\end{figure}
		
		\begin{figure}[H]
			\centering
			\includegraphics[width=0.7\columnwidth]{images/p3/s10}
			\caption{Output of the {\tt sha1sum modifiedhashfile} command} \label{fig:sha1sum2}
		\end{figure}
		
		\begin{figure}[H]
			\centering
			\includegraphics[width=0.7\columnwidth]{images/p3/s11}
			\caption{Output of the {\tt sha256sum modifiedhashfile} command} \label{fig:sha256sum2}
		\end{figure}
		
		\begin{figure}[H]
			\centering
			\includegraphics[width=0.7\columnwidth]{images/p3/s12}
			\caption{Output of the {\tt sha512sum modifiedhashfile} command} \label{fig:sha512sum2}
		\end{figure}
		
		\item Comparing the output from one algorithm in Table~\ref{tab:hash-digest-originalhashfile} and Table~\ref{tab:hash-digest-modifiedhashfile}, is there a difference between the hash digests? If yes, why?
		
		\begin{answer}
			Let's compare the hash digest of {\tt SHA1} algorithm for the two files.
			\begin{enumerate}
				\item {\tt originalhashfile} - {\tt e64870421f78773bf65437f515fd384170ce8839}
				\item {\tt modifiedhashfile} - {\tt b002d4e2319a4bce10a19779bcd37b6829c4ee1a}
			\end{enumerate}
		
			As it can be observed, 
			\begin{itemize}
				\item 	In terms of the length there is no difference between the two hash digests digests (a property of the hashing algorithms), even though we have two different input files.
				
				\item In terms of the similarity there is a drastic change between the two hash digests (the avalanche effect), even though we made only a simple change to the input file.
			\end{itemize}

		\end{answer}
		
		
	\end{enumerate}
\end{document}
