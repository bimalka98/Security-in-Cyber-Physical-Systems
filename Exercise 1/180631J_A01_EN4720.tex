\documentclass[11pt,letterpaper]{article}

\usepackage[letterpaper,margin=0.8in,nohead]{geometry}

\usepackage[colorlinks]{hyperref}
\usepackage{url}
\usepackage{breakurl}

\hypersetup{
    colorlinks,
    linkcolor={red},
    citecolor={red},
    urlcolor={blue}
}

\usepackage{verbatim}
\usepackage{fancyvrb}
\usepackage{scrextend}
\usepackage{enumitem}
\usepackage{url}
\usepackage{tabularx}

\usepackage{caption}
\usepackage{graphicx}
\usepackage{subcaption}

\usepackage{changepage}   % for the adjustwidth environment

\newenvironment{answer}{\em \color{blue} \begin{adjustwidth}{1cm}{1cm}}{\end{adjustwidth}}

% math
\usepackage{amsthm,amsmath}
\usepackage{amsfonts}

\newcommand{\mc}[1]{\mathcal{#1}}	% Mechanisms / Algorithms
\newcommand{\rv}[1]{\mathbf{#1}}    % Random variable

\newcommand{\pr}[1]{\mathrm{Pr}\{#1\}} % Probability

\newtheorem{corollary}{\bf Corollary}%[theorem]
\newtheorem{lemma}{\bf Lemma}%[theorem]
\newtheorem{definition}{\bf Definition}%[section]

\newtheorem{observation}{\bf Observation}%[theorem]

% load cleveref last!
\usepackage[capitalise]{cleveref}


\begin{document}

\title{EN4720: Security in Cyber-Physical Systems \\Exercise --- Big Security Breaches and Exploring CVE}

%% This is an individual assignment!!
%% TODO: put your name and index number here here!
\author{ \textcolor{blue}{Name: Thalagala B. P.} \\ \textcolor{blue}{Index No: 180631J}}

\maketitle

\begin{center}
	\color{red}\bf This is an individual exercise! \\ Due Date: 17 March 2023 by 11.59 PM
\end{center}

\section*{Big Security Breaches}
%

It is important that you keep yourself up to date on previous and contemporary computer security breaches. Find real-world examples of breaches of Confidentiality, Integrity, Availability, Authentication, Authorization, and Non-repudiation using the Internet and fill the following table. Add a three-four sentence explanation for each example.

You can refer to books, web pages, research publications to gather the information. Feel free to copy-paste from the source, but make sure you add the citation. The first row is filled for you as an example.

\begin{table}[htbp]
    \caption{Real-world examples of security breaches.
    }
    \begin{tabularx}{\columnwidth}{|p{3cm}|p{3cm}|X|}
        \hline
        \textbf{Security Goal} & \textbf{Example} & \textbf{Explanation}  \\
        
        \hline
        \textcolor{blue}{Confidentiality} & 
        \textcolor{blue}{Apache Struts vulnerability} & 
        \textcolor{blue}{An Apache Struts vulnerability allowed hackers to steal data on 143 million Equifax customers~\cite{luszcz2018apache}. Struts is vulnerable to remote command injection attacks through incorrectly parsing an attacker’s invalid Content-Type HTTP header. The Struts vulnerability allows these commands to be executed under the privileges of the Web server. This resulted in sensitive data leakage~\cite{apachestruts}.}
        \\ \hline
        
        \hline
        Confidentiality & 
        &
        \\ \hline
        
        \hline
        Integrity & 
        &
        \\ \hline
        
        \hline
        Availability & 
        &
        \\ \hline
        
        \hline
        Authentication & 
        &
        \\ \hline
        
        \hline
        Authorization & 
        &
        \\ \hline
        
        \hline
        Non-Repudiation &
        &
        \\ \hline
        
        
    \end{tabularx}
\end{table}

The following sites may help you to get started:
\begin{itemize}
    \item http://www.networkworld.com/topics/security.html
    \item http://www.zdnet.com.au/topic/security/ 
\end{itemize}

Furthermore, as security professionals, it is important that we stay updated. Below are some resources that you can use to stay updated. 

\begin{table}[htbp]
    \caption{Channels to stay informed.
    }
    \begin{tabularx}{\columnwidth}{|X|X|X|X|}
        \hline
        \textbf{Technology Partners} & \textbf{Government}       & \textbf{Security Organizations} & \textbf{Security News Sites}  \\
        
        \hline
        Microsoft & 
        US-CERT & 
        SANS ISC &
        Dark Reading
        \\ \hline
        
        \hline
        Red Hat & 
        NIST NVD & 
        &
        The Hacker News
        \\ \hline
        
        \hline
        Ubuntu & 
        SLCERT &
        &
        CSO Online
        \\ \hline
    \end{tabularx}
\end{table}

%%%%%%%%%%%%%%%%%%%%

\section*{Exploring CVE}
%

CVE, short for Common Vulnerabilities and Exposures, is a list of publicly disclosed computer security flaws. Learn more about CVE \href{https://www.redhat.com/en/topics/security/what-is-cve}{here}. 

Search the CVE database at \href{https://cve.mitre.org/}{cve.mitre.org} for vulnerabilities in one of the smartphone apps you use. Study a few of them carefully to get a sense of how beneficial this database can be for a security professional. Identify five flaws in your selected app and fill out the table below. 

\begin{itemize}
    \item Column 1: CVE ID of the vulnerability.
    \item Column 2: A brief description of the vulnerability in a way that a novice user can understand.
    \item Column 3: Which security goal (out of the CIA triad) is breached as a result of the vulnerability.
    \item Column 4: Add the title and URL for any known real-life incidents.
\end{itemize}

\begin{table}[htbp]
    \caption{Vulnerabilities in a smartphone application.
    }
    \begin{tabularx}{\columnwidth}{|X|X|p{3cm}|X|}
        \hline
        \textbf{Vulnerability} & \textbf{Brief Description} & \textbf{Breach of security goal} & \textbf{Any known real-life case with URL}\\
        
        \hline
        & 
        &
        &
        \\ \hline
        
        & 
        &
        &
        \\ \hline
        
        &
        &
        &
        \\\hline
        
        &
        &
        &
        \\\hline
        
        &
        &
        &
        \\\hline
    \end{tabularx}
\end{table}


\bibliographystyle{plain} % We choose the &quot;plain&quot; reference style
\bibliography{refs} % Entries are in the &quot;refs.bib&quot; file</code></pre>

\end{document}
