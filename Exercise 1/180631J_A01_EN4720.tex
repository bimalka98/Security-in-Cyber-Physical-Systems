\documentclass[11pt,letterpaper]{article}

\usepackage[letterpaper,margin=0.8in,nohead]{geometry}

\usepackage[colorlinks]{hyperref}
\usepackage{url}
\usepackage{breakurl}

\hypersetup{
    colorlinks,
    linkcolor={red},
    citecolor={red},
    urlcolor={blue}
}

\usepackage{verbatim}
\usepackage{fancyvrb}
\usepackage{scrextend}
\usepackage{enumitem}
\usepackage{url}
\usepackage{tabularx}

\usepackage{caption}
\usepackage{graphicx}
\usepackage{subcaption}

\usepackage{changepage}   % for the adjustwidth environment

\newenvironment{answer}{\em \color{blue} \begin{adjustwidth}{1cm}{1cm}}{\end{adjustwidth}}

% math
\usepackage{amsthm,amsmath}
\usepackage{amsfonts}

\newcommand{\mc}[1]{\mathcal{#1}}	% Mechanisms / Algorithms
\newcommand{\rv}[1]{\mathbf{#1}}    % Random variable

\newcommand{\pr}[1]{\mathrm{Pr}\{#1\}} % Probability

\newtheorem{corollary}{\bf Corollary}%[theorem]
\newtheorem{lemma}{\bf Lemma}%[theorem]
\newtheorem{definition}{\bf Definition}%[section]

\newtheorem{observation}{\bf Observation}%[theorem]

% load cleveref last!
\usepackage[capitalise]{cleveref}


\begin{document}

\title{EN4720: Security in Cyber-Physical Systems \\Exercise --- Big Security Breaches and Exploring CVE}

%% This is an individual assignment!!
%% TODO: put your name and index number here here!
\author{ \textcolor{blue}{Name: Thalagala B. P.} \\ \textcolor{blue}{Index No: 180631J}}

\maketitle

\begin{center}
	\color{red}\bf This is an individual exercise! \\ Due Date: 17 March 2023 by 11.59 PM
\end{center}

\section*{Big Security Breaches}
%

It is important that you keep yourself up to date on previous and contemporary computer security breaches. Find real-world examples of breaches of Confidentiality, Integrity, Availability, Authentication, Authorization, and Non-repudiation using the Internet and fill the following table. Add a three-four sentence explanation for each example.

You can refer to books, web pages, research publications to gather the information. Feel free to copy-paste from the source, but make sure you add the citation. The first row is filled for you as an example.

\begin{table}[htbp]
    \caption{Real-world examples of security breaches.
    }
    \begin{tabularx}{\columnwidth}{|p{3.4cm}|p{3.4cm}|X|}
        \hline
        \textbf{Security Goal} & \textbf{Example} & \textbf{Explanation}  \\
        
        \hline
        \textcolor{blue}{Confidentiality} & 
        \textcolor{blue}{Apache Struts vulnerability} & 
        \textcolor{blue}{An Apache Struts vulnerability allowed hackers to steal data on 143 million Equifax customers~\cite{luszcz2018apache}. Struts is vulnerable to remote command injection attacks through incorrectly parsing an attacker’s invalid Content-Type HTTP header. The Struts vulnerability allows these commands to be executed under the privileges of the Web server. This resulted in sensitive data leakage~\cite{apachestruts}.}
        \\ \hline
        
        \hline
        \textbf{Confidentiality} (\textit{Keep data private or secret/ Control access}) & Marriott hotel chain's reservation system compromisation 2018
        & In late 2018, the Marriott hotel chain announced that one of its reservation systems had been compromised, with hundreds of millions of customer records, including credit card and passport numbers, being exfiltrated by the attackers\cite{fruhlinger_marriott_2020}.
        \\ \hline
        
        \hline
        \textbf{Integrity} (\textit{Trusted/ No unauthorized modification}) & 
        SolarWinds attack of 2020
        &  SolarWinds offers an IT performance management and monitoring system called Orion. The hackers used a supply chain attack to insert malicious pieces of code into the Orion framework. It allowed the hackers to access system files and hide their tracks by blending into the Orion activity. More than 18,000 customers of SolarWinds were affected including  the US departments of health, treasury, and state\cite{noauthor_solarwinds_nodate}.
        \\ \hline
        
        \hline
        \textbf{Availability} (\textit{Systems are up and running}) &
        GitHub Distributed Denial-of-Service (DDoS) attack 2018 &
        February 28, 2018 GitHub.com was unavailable from 17:21 to 17:26 UTC and intermittently unavailable from 17:26 to 17:30 UTC due to a DDoS attack. The attack originated from over a thousand different autonomous systems (ASNs) across tens of thousands of unique endpoints. It was an amplification attack using the Memcached-based approach that peaked at 1.35Tbps via 126.9 million packets per second\cite{kottler_february_2018}.
        \\ \hline
        
        \hline
        \textbf{Authentication} (\textit{Who are you?/ Are you who you claim to be?}) & 2012 LinkedIn hack
        & The social networking website LinkedIn was hacked on June 5, 2012, and passwords for nearly 6.5 million user accounts were stolen by Russian cyber criminals. Owners of the hacked accounts were no longer able to access their accounts, and the website repeatedly encouraged its users to change their passwords after the incident\cite{LinkedIn}.
        \\ \hline
        
        \hline
        \textbf{Authorization} (\textit{What abilities and access should this user have?}) & Adobe Systems Data Breach 2013
        & October 2013, hackers stole login information and nearly 3 million credit card numbers from 38 million Adobe users\cite{Adobe}.
        \\ \hline
        
        \hline
        \textbf{Non-Repudiation} (\textit{To not allow someone to deny something}) &
        2016 Democratic National Committee (DNC) email leak
        & A collection of DNC emails stolen by one or more hackers, who are alleged to be Russian intelligence agency hackers. This collection included 19,252 emails and 8,034 attachments from the DNC, the governing body of the United States' Democratic Party\cite{dnc_mails}.
        \\ \hline
        
        
    \end{tabularx}
\end{table}

The following sites may help you to get started:
\begin{itemize}
    \item http://www.networkworld.com/topics/security.html
    \item http://www.zdnet.com.au/topic/security/ 
\end{itemize}
%
%Furthermore, as security professionals, it is important that we stay updated. Below are some resources that you can use to stay updated. 
%
%\begin{table}[htbp]
%    \caption{Channels to stay informed.
%    }
%    \begin{tabularx}{\columnwidth}{|X|X|X|X|}
%        \hline
%        \textbf{Technology Partners} & \textbf{Government}       & \textbf{Security Organizations} & \textbf{Security News Sites}  \\
%        
%        \hline
%        Microsoft & 
%        US-CERT & 
%        SANS ISC &
%        Dark Reading
%        \\ \hline
%        
%        \hline
%        Red Hat & 
%        NIST NVD & 
%        &
%        The Hacker News
%        \\ \hline
%        
%        \hline
%        Ubuntu & 
%        SLCERT &
%        &
%        CSO Online
%        \\ \hline
%    \end{tabularx}
%\end{table}

%%%%%%%%%%%%%%%%%%%%

\section*{Exploring CVE}
%

CVE, short for Common Vulnerabilities and Exposures, is a list of publicly disclosed computer security flaws. Learn more about CVE \href{https://www.redhat.com/en/topics/security/what-is-cve}{here}. 

Search the CVE database at \href{https://cve.mitre.org/}{cve.mitre.org} for vulnerabilities in one of the smartphone apps you use. Study a few of them carefully to get a sense of how beneficial this database can be for a security professional. Identify five flaws in your selected app and fill out the table below. 

\begin{itemize}
    \item Column 1: CVE ID of the vulnerability.
    \item Column 2: A brief description of the vulnerability in a way that a novice user can understand.
    \item Column 3: Which security goal (out of the CIA triad) is breached as a result of the vulnerability.
    \item Column 4: Add the title and URL for any known real-life incidents.
\end{itemize}

\begin{table}[htbp]
    \caption{Vulnerabilities in a smartphone application.
    }
    \begin{tabularx}{\columnwidth}{|X|p{8cm}|p{3cm}|X|}
        \hline
        \textbf{Vulnerability} & \textbf{Brief Description} & \textbf{Breach of security goal} & \textbf{Any known real-life case with URL}\\
        
        \hline
        CVE-2020-1908
        & Improper authorization of the Screen Lock feature in WhatsApp and WhatsApp Business for iOS prior to v2.20.100 could have permitted use of `Siri' to interact with the WhatsApp application even after the phone was locked. An attacker who gains access to Siri may be able to read and send messages, access contacts, and perform other actions within the WhatsApp application without having to unlock the phone.
        & Confidentiality
        & None reported
        
        \\ \hline
        CVE-2021-24035
        & A lack of filename validation when unzipping archives prior to WhatsApp for Android v2.21.8.13 and WhatsApp Business for Android v2.21.8.13 could have allowed path traversal attacks that overwrite WhatsApp files. This can allow an attacker to access files and directories outside of the intended location by manipulating the path used to access them. 
        & Integrity
        & None reported
        
        \\ \hline
        CVE-2020-1901
        & Receiving a large text message containing URLs in WhatsApp for iOS prior to v2.20.91.4 could have caused the application to freeze while processing the message. This can result in a denial of service to the user when he is involved in some interaction with the App.
        & Availability
        & None reported
        
        \\\hline
        CVE-2019-3571
        & An input validation issue affected WhatsApp Desktop versions prior to 0.3.3793 which allows malicious clients to send files to users that would be displayed with a wrong extension. The attacker can send files that can either execute malicious code, access sensitive information or both. The user may believe that they are opening a safe file, but the actual contents could be malicious.
        & Integrity, Confidentiality
        & None reported
        \\\hline
        CVE-2022-36934	
        & An integer overflow in WhatsApp could result in remote code execution in an established video call. The attacker could gain access to the system, potentially steal sensitive information or modify the system's behavior, leading to a loss of data or disruption of service.
        & Integrity, Confidentiality
        & None reported
        \\\hline
    \end{tabularx}
\end{table}


\bibliographystyle{plain} % We choose the &quot;plain&quot; reference style
\bibliography{refs} % Entries are in the &quot;refs.bib&quot; file</code></pre>

\end{document}
